%!TEX TS-program = xelatex
%!TEX encoding = UTF-8 Unicode
\documentclass[letterpaper,12pt]{article}

% == Metadata =============================================
\def\ptitle{من و فلسطین}
\def\pauthor{عزیز آشفته‌برگی}
\title{\ptitle}
\author{\pauthor}
\date{\today}

% == Packages =============================================
\usepackage[parfill]{parskip}      % Begin paragraphs with an empty line rather than an indent
\usepackage{fancyhdr}                    % Running Headers and footers
\usepackage{graphicx}                    % Images loaded externally
\usepackage[usenames,dvipsnames]{xcolor} % Color setup
\usepackage[pdftex]{hyperref}            % Making links in PDF
\usepackage[localise]{xepersian} 		     % Persian in TEX

% == Package Settings ========================================
\settextfont[Scale=1]{XB Kayhan}
\setlatintextfont[Scale=1]{Helvetica}
\setdigitfont[Scale=1]{XB Zar}
%\defpersianfont\Nastaliq[Scale=1]{IranNastaliq}
%\deflatinfont\Myriad[Scale=1]{Myriad Pro}


%-- Hyperref -------------------------------------------------------(fold)
\hypersetup{
    bookmarks=true,          % show bookmarks bar?
    unicode=true,            % non-Latin characters in Acrobat’s bookmarks
    pdftoolbar=true,         % show Acrobat’s toolbar?
    pdfmenubar=true,         % show Acrobat’s menu?
    pdffitwindow=false,      % window fit to page when opened
    pdfstartview={FitH},     % fits the width of the page to the window
    pdftitle={\ptitle},      % title
    pdfauthor={\pauthor},    % author
    pdfcreator={XelaTeX},    % creator of the document
    pdfproducer={\pauthor},  % producer of the document
    pdfnewwindow=false,      % links in new window
    colorlinks=true,         % false: boxed links; true: colored links
    linkcolor=BrickRed,      % color of internal links
    citecolor=OliveGreen,    % color of links to bibliography
    filecolor=magenta,       % color of file links
    urlcolor=BlueViolet      % color of external links
}
% -------------------------------------------------------------------(end)
%-- Fancy Header ---------------------------------------------------(fold)
\pagestyle{fancy}
%\renewcommand{\chaptermark}[1]{\markboth{#1}{}}
\renewcommand{\sectionmark}[1]{\markright{\thesection\ #1}}
\fancyhf{}                              % delete current header and footer
\fancyhead[LE,RO]{\bfseries\thepage}
\fancyhead[LO]{\bfseries\rightmark}
\fancyhead[RE]{\bfseries\leftmark}
\renewcommand{\headrulewidth}{0.5pt}
\renewcommand{\footrulewidth}{0pt}
\addtolength{\headheight}{0.5pt}        % space for the rule
\fancypagestyle{plain}{
  \fancyhead{}                          % get rid of headers on plain pages
  \renewcommand{\headrulewidth}{0pt}    % and the line
}
% -------------------------------------------------------------------(end)

\begin{document}
\maketitle
\begin{latin} 
Latin 
\end{latin} 
\begin{persian} 
فارسی  
\end{persian}
% == Document ============================================
\بخش{آشنایی}
سا‌ل‌های سال نوار غزه و کرانه باختری  کلماتی بود که ماهی چندبار حیاتی و بابان در اخبار برایم تکرار می‌کردند تا یادم بماند در غربی ترین نقطه خاورمیانه مظلومی سپید جامه در مقابل ظالمی سیاه‌دل دست و پنجه نرم‌ می‌کند. انتفاضه کلمه‌ای عربی بود که بارها می‌شنیدم و می‌خواندم اما هیچ‌گاه اشتیاقی برای کشف معنی‌اش نداشتم، کلمه‌ای که سنگ پرانی چند جوان چفیه‌به‌سرپیچیده و گاز اشک‌آور را در ذهنم متبادر می‌کرد و میدان فلسطین میدانی بود که همه شهر‌ها حداقل یکی به این نام داشتند آن هم در شلوغ‌ترین و پرترافیک‌ترین نقطه شهر.

در تمام آن سال‌ها فلسطین با تمام اخبار و وقایعش جزیی از زندگی روزمره من بود، بدون اینکه ذره‌ای کنجکاوی‌ام را تحریک کند یا به فکر وادارد. فلسطین مساله‌ای حل شده در زندگی من و هم‌سن و سال‌هایم بود. جایی که خیر و شر، سیاهی و سپیدی در جنگند.

حتی وقتی برای ادامه تحصیل از ایران خارج‌ می‌شدم و تا حدی به مسایل سیاسی بین‌الملل علاقه‌مند شده بودم فلسطین آخرین جایی بود که می‌توانست توجه مرا به خود جلب کند. اما داستان زندگیم قرار بود طور دیگری رقم بخورد.


در همان روزهای ابتدایی که سخت به دنبال جایی برای سکونت بودم، یکی از هم‌دانشکده‌ای‌ها شماره تلفن مردی را داد که می‌خواست خانه اجاره‌ای‌اش را با دیگری شریک شود. طبق معمول زنگ زدم و قرار گذاشتم. مرد پشت تلفن انگلیسی را با لهجه اما روان صحبت می‌کرد و نامش ملک بود. از آشنای مشترکمان ملیتش را که پرسیده بودم گفته بود که نمی‌داند ولی حدس می‌زد که فرانسوی باشد. ممکن بود برای دوست مشترکمان نام ملک فرانسوی به نظر برسد اما برای من مسلمان‌زاده‌ی عربی‌خوانده قطعا فرانسوی نبود. در ذهنم او را عربی سیه‌چرده از مهاجران لیبیایی فرانسه فرض کردم و منتظر قرار فردا ظهر ماندم.

خانه ملک آپارتمانی نوساز در منطقه‌ای جنگلی ولی دور از دانشگاه بود. از تراموا که پیاده‌شدم دوباره زنگ زدم و گفت که در ایستگاه منتظرم ایستاده ولی من هر چه سر می‌گرداندم چیزی شبیه آنچه در ذهن داشتم نمی‌دیدم. ایستگاه تقریبا خلوت بود و تنها کسی که گوشی موبایلی نزدیک گوش داشت جوانی سفید پوست بود که پالتویی مشکی بر تن و سر وضع بسیار مرتبی داشت. با تعجب جلو رفتم و احوال‌پرسی کردیم. بلافاصله به دوست مشترکمان حق دادم، در نگاه اول او یک فرانسوی بود. پوستی سفید با دماغی بزرگ و استخوانی؛ ابرو‌هایی سیاه و پرپشت که با فاصله از بالای فریم سیاه عینکش پیشانیش را بلندتر از آنچه بود نشان می‌داد؛ و بالاخره و لباس‌های موقر و به ظاهر گران‌قیمت از او یک فرانسوی کلاسیک تمام عیار‌ می‌ساخت. در راه سوال‌های معمول را می‌پرسید چند وقت است که آمده‌ام؟، چه می‌خوانم؟ و چند‌سالم است؟ و نزدیک در ورودی بودیم که ملیتم را پرسید. گفتم که ایرانیم. ابرو‌هایش روی پیشانی‌ بلندش بالاتر رفت و لبخدی زد و گفت پس مسلمانی؟ تایید کردم. در را که باز می‌کرد من اولین سوال را کردم: تو از کجا آمدی؟ گفت: فلسطین. بهت زده‌بودم، فلسطین در لیست صد و اندی کشور که حدس می‌زدم ملک از آنجا آمده باشد اگر آخری نبود قطعا در بین بیست کشور آخر بود.

خانه را دیدم. دوست داشتنی بود، ملک هم آدم خوبی به نظر می‌آمد تصمیمم را گرفتم و همان شب من و ملک رسما برای یکسال هم‌خانه شدیم. و این گونه فلسطین دوباره وارد زندگی روزمره من شد… اما این بار از نوعی دیگر…

این روزها همه از فلسطین می‌گویند و باز داستان تراژدیک فلسطین و مردم نوار غزه ورد زبان‌هاست. الحق هم باید از فلسطین و اسراییل و کشمکش بحران خاورمیانه نوشت و حرف زد، ولی نه از جنس آن نوشته‌های دلسوزانه و همدردمابانه که ریشه در دل‌سنگی و بی‌دردی دارد. من خسته‌ام از تکرار مکرر دلسوزی ساده‌لوحانه غربی‌ها برای مادر اسراییلی و خسته‌تر از همدردی و حمایت کورکورانه جوامع مسلمان از دیگر مادری فلسطینی. هیچ کسی نیازمند همدردی نیست. هیچ کسی نیازمند دلسوزی نیست. دلسوزی و همدردی زایده آن نگاه یک سو نگر و خود حق پندار به این تراژدی تاریخی است.
قصدم قصه گفتن نیست. می‌خواستم از تجربه خودم در مورد فلسطین بگویم ولی حیفم آمد شما را همراه آن لحظه‌های یک‌باربرای‌همیشه آغازین این تجربه نکنم. در ادامه سعی می‌کنم از آن شب‌های بلند و حرف‌های بی‌انتهای من و ملک در طبقه چهارم آن آپارتمان بگویم تا شاید شما را هم در آنچه این روزها به آن فکر می‌کنم شریک کنم. واقعا ما در مورد فلسطین چقدر می‌دانیم؟

\بخش{دوستی}
پنچره‌های خانه به پشت ساختمان باز می‌شد که نمایی از زمینی مسطح و پوشیده از چمن داشت. از اطراف با تپه‌های جنگلی محدود شده بود و دورتر ها به اتوبانی می‌رسید. دورتادور زمین سیمکشی شده بود و چند کامیون و جیپ نظامی‌ کهنه — از آن‌ها که شاید میراث جنگ جهانی دوم بود — بی نظم در چند گوشه به چشم می‌خورد. معلوم بود سال‌هاست که خاک می‌خوردند.

در نزدیکی خانه موزه جنگ بود. ساختمانی باشکوه با معماری ویکتوریایی که تنها نشانه‌هایش از جنگ توپ‌های سنگی قدیمی پشت پرچین‌ها و در‌های همیشه بسته‌اش بود. موزه جنگ هم مثل کامیون‌های نظامی و مثل خاطرات کودکی من از جنگ، مسکوت و رها شده بود. اما نه برای ملک. می‌گفت که همان روزهای اول برای بازدید رفته اما موزه تنها دو روز در هفته آن هم برای سه ساعت باز است. عکس‌های زیادی که از ساختمان و محوطه بیرونی گرفته بود را نشانم می‌داد با اشتیاق برایم توضیح می‌داد. قرار شد روزی با هم برای بازدید برویم، اما من هیچ‌وقت به موزه جنگ نرفتم.

ملک آدم کم حرفی بود. در روستایی به نام میثلون در شرق جنین از شهرهای بزرگ کرانه باختری بزرگ شده بود و در دانشگاه جنین معماری خوانده و برای ادامه تحصیل به اینجا آمده بود. آدم غمگینی نبود اما شاد هم نبود. وقتی حرفی می‌زدم که قبول نداشت می‌خندید و بعد با آرامش و اعتماد به نفس، با لبخندی بر لب، جواب می‌داد. اعتماد به نفسی داشت از جنس آن اعتماد به نفس‌هایی که تنها ایمان مذهبی قوی در آدمی ایجاد می‌کند.

اختلاف سنی‌مان زیاد نبود ولی او پیر و شکسته شده بود و من جوان‌تر از آنچه تاریخ تولدم نشان‌ می‌داد. همان شب‌های اول بود که برای باز کردن سر صحبت سنش را پرسیدم، جواب داد ۳۰ گفتم که باور نمی‌کنم. خندید و گفت: «چه عیبی دارد وقتی تو سنت را بیشتر از آنچه هست می‌گویی من هم کمی کمترش کنم؟» خندیدم و اصرار کردم که راست می‌گویم، قرار شد پاسپورت‌هایمان را سند صداقتمان کنیم. رفت و از جیب پالتو‌اش دو پاسپورت درآورد. با عجله‌ و شوقی کودکانه گفتم: «دو تا پاسپورت داری؟ خوش به حالت. حتما یکی اسراییلی است؟» خندید و گفت: «نه یکی اردنی است و دیگری فلسطینی. جفت‌شان هم به هیچ دردی نمی‌خورند، به هیچ‌جا نمی‌توانی با این‌ها سفر کنی». از عجولانه حرف زدنم پشیمان شدم، خواستم همدردی کنم و گفتم لااقل به اینجا می‌شود سفر کرد. ناگهان چیزی به ذهنم رسید پاسپورتم را به دستش دادم. گفتم «ما هم جایی است که با این پاسپورت نمی‌توانیم به آن سفر کنیم، صفحه آخر نوشته است». پاسپورت را باز کرد، صفحه آخر زیر سه تذکر به زبان فارسی خوشبختانه سومیش را به انگلیسی ترجمه کرده:‌‌ «دارنده این گذرنامه حق مسافرت به فلسطین اشغالی را ندارد.» جمله را که خواند سرش را بالا آورد و گفت: «نمی‌دانستم برای رفتن به جهنم اینقدر علاقه داری!»… با هم خندیدیم.

بعدها در مورد پاسپورت اردنی بیشتر حرف زدیم. پس از شکست اعراب در جنگ شش روزه، اسراییل اعلام می‌کند که هیچ‌ مسوولیتی در مقابل اعراب ساکن کرانه باختری و نوار غزه ندارد و دول اردن و مصر را مسوول رسیدگی و مدیریت اعراب این مناطق می‌داند. و این‌گونه فلسطینی‌ها عملا فرزند‌خوانده‌های ناخواسته این دو کشور می‌شوند. درک احساس یک فلسطینی در این لحظه کار سختی نباید باشد. دشمنی که تو را مرده می‌خواهد و پدرخوانده‌ای که حاضر نیست زنده تو را در خانه‌اش تحمل کند. شاه حسین پادشاه وقت اردن برای رهایی از دست این فرزند‌خوانده‌ها تصمیم می‌گیرد آنها را به مهاجرت تشویق کند و برای این‌کار برای‌ آنها پاسپورت‌های خاص اردنی (متفاوت با آنچه شهروندان اردنی دارند) صادر می‌کند. در این میان کسی شادمان‌تر از دولتمردان اسراییل پیدا نمی‌شود. این دقیقا هما‌ن‌کاری بود که سال‌ها آنها ‌می‌خواستند بکنند ولی نتوانسته بودند. آنها برای کمک به این راه‌حل هوشمندانه شاه‌حسین با فشار به روسیه و امریکا و خصوصا دولت‌های اروپایی آنها را متقاعد می‌کنند که هر یک تعدادی از این فرزندان ناخواسته را قیم شوند. مضحک آنکه درست پس از امضای قرارداد کمپ دیوید بین عرفات و اسحاق رابین ناپدری اردنی به بهانه اینکه دیگر فلسطین دولتی خود‌گردان و مستقل دارد از صدور و تمدید گذر‌نامه‌ها خود‌داری می‌کند.


\end{document}